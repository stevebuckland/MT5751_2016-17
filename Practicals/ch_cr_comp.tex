\chapter{Capture-Recapture\label{ch:cr_comp}}


This document is to help you get started with the \verb|R| package \verb|secr|. The package has extensive help, which you should use. In particular, you should read the vignettes \verb|secr.overview.pdf| and \verb|secr-datainput.pdf| to help you get going. You can get these by typing 

\begin{verbatim}
library(secr)
?secr
\end{verbatim}
in your R window and then scrolling down in the help window that this opens, until you find links to these documents.\\

We will play with the \verb|stoatDNA| dataset to get used to \verb|secr|.

\begin{verbatim}
data(stoatDNA)   # get stoat hair snare dataset
?stoatDNA        # find out something about it
summary(stoatCH) # look at the capture histories
plot(stoatCH)    # plot it
\end{verbatim}

What kind of detectors were used (single-catch traps, multi-catch traps, binary detectors or count detectors)?

How many capture occasions were there?

How many stoats were detected over the whole survey?

How many stoats were captures once, twice, three times, ...?

Extract the trap data from \verb|stoatCH| and plot it:

\begin{verbatim}
traps=traps(stoatCH)
plot(traps)
\end{verbatim}

Now fit a model to the data:

\begin{verbatim}
stoat.model.HN=secr.fit(stoatCH, buffer = 1000, detectfn = 0)
\end{verbatim}

Try that again, with the additional argument \verb|print.level=0| (which is an argument of the optimiser \verb|nlm| that \verb|secr.fit| uses to find the maximum of the likelihood function):

\begin{verbatim}
stoat.model.HN <- secr.fit(stoatCH, buffer = 1000, detectfn = 0,trace=0)
\end{verbatim}
Do you see what changed while fitting the model with this extra argument?\\

Now look at the output from the fit:
\begin{verbatim}
stoat.model.HN       # look at the estimates
\end{verbatim}

Verify that you understand what the ``\verb|Beta parameters|'' and ``\verb|Fitted (real) parameters|'' are by calculating the latter manually from the former.\\

Plot the detection function:
\begin{verbatim}
plot(stoat.model.HN,xval=0:1000,sigmatick=TRUE,limits=TRUE,ylim=c(0,0.12)) 
\end{verbatim}

Do you understand what the solid, dashed and vertical lines are? See the help for \verb|plot.secr| if you do not. \\

Now try fitting models with some other detection functions. (See the \verb|secr| help pages on \verb|detectfn| for the options available to you - don't try them all, there are way too many!)\\

Try fitting models of type M$t$, M$b$, M$h$ for $\sigma$ and/or $g_0$. (Table 5 in the vignette \verb|secr.overview.pdf| is useful here - but only consider \verb|t|, \verb|T|, \verb|b|, \verb|B| and \verb|h2|.)

